%! Author = tomas.bahnik
%! Date = 11/16/2021

% Preamble
\documentclass[12pt]{article}

% Packages
\usepackage[total={7in,10in}]{geometry}
\usepackage{amsmath}
\geometry{a4paper}


% Document
\begin{document}

    \section{Konvergence alternující řady}\label{sec:linalg}

    \begin{equation}
        \label{eq:rada}
        \sum_{n = 1}^{\infty}\frac{(-1)^{n-1}}{n^2 -6 n + 10}
    \end{equation}

    Platí=li pro řadu~\eqref{eq:rada}
    \[a_1 - a_2 + a_3 -a_4 \dots a_n \geq 0\]

    následující podmínky

    \begin{equation}
        \label{eq:podminky}
        a_1\geq a_2 \geq a_3 \geq \dots, \lim_{n \to \infty} a_n = 0
    \end{equation}

    je řada konvergentní.

    V případě řady~\eqref{eq:rada} platí podminky~\eqref{eq:podminky}

    \[a_1 = \frac{1}{5}, a_2 = -\frac{1}{2}, a_3 = 1, a_4=-\frac{1}{2}, a_5=\frac{1}{5}, a_6 = -\frac{1}{10}\]

    počínaje $a_3$ včetně. Součet

    \begin{equation}\label{eq:rada3}
        \sum_{n = 1}^{n=2}\frac{(-1)^{n-1}}{n^2 -6 n + 10} = -\frac{3}{10}
    \end{equation}

    je konečné číslo a na konvergenci řady~\eqref{eq:rada} nic nemění

\end{document}