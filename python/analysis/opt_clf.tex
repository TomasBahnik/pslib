%! Author = tomas.bahnik
%! Date = 11/16/2021

% Preamble
\documentclass[11pt]{article}

% Packages
\usepackage[total={7in,10in}]{geometry}
\usepackage{amsmath}
\usepackage{graphicx}
\usepackage[colorlinks = true,
    linkcolor = blue,
    urlcolor  = blue,
    citecolor = blue,
    anchorcolor = blue]{hyperref}
\geometry{a4paper}


% Document
\begin{document}

    \section{Optimální binární klasifikátor}\label{sec:opt_clf}
    Optimální hodnota parametru  $\alpha$, pro daný klasifikátor,
    byla určena z podmínky maximální správnosti (accuracy) t.j. poměru počtu správně
    klasifikovaných vzorků vs. počet všech vzorků viz.
    \href{https://en.wikipedia.org/wiki/Receiver_operating_characteristic}{Receiver operating characteristic}

    \begin{equation}
        ACC = \frac{TP + TN}{P + N} = \frac{TP + TN}{TP + TN + FP + FN}\label{eq:accuracy}
    \end{equation}

    Volba klasifikátoru pak vychází z požadavku {\em hlavně bezpečně} t.j.\
    minimalizovat případy neautorizovaného přístupu, kdy platí že $y_{true}=0$ ale $y_{pred}=1$,
    což odpovídá minimální false positive rate

    \begin{equation}
        FPR = \frac{FP}{N}=\frac{FP}{TN + FP}\label{eq:fpr}
    \end{equation}

    \begin{center}
        \begin{tabular}{ c c }
            TN(0,0) & FP(0,1) \\
            FN(1,0) & TP(1,1) \\
        \end{tabular}
    \end{center}

    Matice záměn a ROC křivky pro klasifikátory $C1..C5$ ukazují $C1$ s $\alpha_{22}$ jako
    optimální klasifikátor

    \includegraphics[scale=0.45]{cm_1}
    \includegraphics[scale=0.45]{cm_2}

    \includegraphics[scale=0.45]{cm_3}
    \includegraphics[scale=0.45]{cm_4}

    \includegraphics[scale=0.45]{cm_5}
    \includegraphics[scale=0.45]{roc}

    Použité funkce
    \texttt{confusion\_matrix, ConfusionMatrixDisplay, roc\_curve, roc\_auc\_score}
    z knihovny \texttt{sklearn.metrics}
\end{document}